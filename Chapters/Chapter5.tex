% Chapter 1
\stepcounter{cap}
%\chapter{cap1}
\label{cap5}

\mychapter{5}{Capitolul \arabic{cap} \\ DESCRIEREA ALGORITMILOR}
%\chapter{\arabic{cap}.Introducere} % Main chapter title

\label{Chapter5} % For referencing the chapter elsewhere, use \ref{Chapter1} 

\thispagestyle{fancy}

%-----------------------------------------------------------------

\section{Învățarea rutelor} 
Rutele sunt învățate printr-un mecanism inteligent. Acest lucru are avantajul de a reduce semnificativ baza de date și de a accelera încărcarea datelor.

	\subsection{Reducerea waypoint-urilor} 
	Criteriile pentru excluderea waypoint-urilor sunt:
	\begin{enumerate}
	 \setlength\itemsep{0em}
		\item Vehiculul nu și-a schimbat orientarea semnificativ (valoare standard: > 15$^{\circ}$)
		\item Pozițiile sunt apropiate una de cealaltă (valoare standard: > 200m)
	\end{enumerate}
	
	Valorile standard pot fi configurate  înaintea procesului de compilare.
	
	
	\subsection{Verificarea rutelor duble} 
	În cazul în care ruta curentă are o secțiune similară cu o rută deja invățată, unitatea software PNavDataManager va detecta secțiune respectivă și va stoca numai waypoint-urile situate după aceasta. În schimb, toate destinațiile sunt memorate. Criteriile pentru a detecta o astfel de rută sunt:
	
		\begin{enumerate}
	 \setlength\itemsep{0em}
		\item Destinația noii rute trebuie să fie situată într-o rază de 1km față de locația vechii destinații
		\item Punctul de start al noii rute trebuie să fie situată într-o rază de 1km față de punctul de plecare vechii destinații
		\item Toate waypoint-urile noii rute trebuie:
				\begin{enumerate}
				 \setlength\itemsep{0em}
					\item Să fie situat într-o anumită rază față de punctul de start al vechii rute
					\item Sau să fie situat într-o anumită rază față de destinația vechii rute
					\item Sau să fie situat într-o anumită rază față de un waypoint al vechii rute
				\end{enumerate}
	\end{enumerate}

Pragurile pot fi configurate înaintea procesului de compilare.


	\subsection{Filtrarea rutelor inutile}
	În timpul învățării, rutele prea scurte (< 2km) nu vor fi stocate deoarece acest lucru ar însemna faptul că utilizatorul este destul de aproape de destinație.
	\vspace{6pt}
  \\Totodata, rutele foarte lungi (> 200km) vor fi de asemenea exclude din stocare deoarece ele nu reprezintă rute uzuale.
	\vspace{6pt}
  \\Pragurile pot fi configurate înaintea procesului de compilare.
	
\section{Eliberarea de spațiu} 
Dacă pragul setat inițial pentru dimensiunea maximă a bazei de date este atins, se va activa funcția de eliberare a spațiului. Aceasta va șterge obiectele cele mai vechi pentru a crea loc pentru obiectele noi.

\section{Furnizarea predicțiilor} 
Pentru ca predicțiile să fie disponibile sunt necesari doi pași.
\vspace{6pt}
\\Primul pas reprezintă încărcarea datelor revelante, în timp de al doilea constă în prioritizarea lor.

	\subsection{Filtrarea datelor}
	Pentru o încărcare selectivă și mai rapidă a datelor din unitatea PNavDataStorage, sunt create interogări. Sunt definiți trei pași în filtrare, unde pasul următor se excută doar în cazul în care cel curent nu a returnat destule rezultate:
		\begin{enumerate}
				 \setlength\itemsep{0em}
					\item Filtrarea rutei în funcție de distanța de la poziția curentă la waypoint-urile rutelor
					\item Filtrarea rutei în funcție de distanța de la poziția curentă la punctele de start ale rutelor
					\item Filtrarea rutelor în funcție de timpul scurs până la ajungerea la destinație
		\end{enumerate}

	Toate rutele ce duc la o destinație situată la o distanță mai mică de 1km față de poziția actuală sunt ignorate deoarece utilizatorul aproape a ajuns la eventuala destinație.
	\vspace{6pt}
	\\În cazul predicțiilor baza pe timp, modulul furnizează o predicție fără a cunoaște ruta, ci doar destinația sa. Un astfel de caz ar fi cel în care utilizatorul a condus pe o rută în intervalul luni-miercuri, însă in ziua de joi a pornit de la o altă locație. Astfel, bazat pe timp, modulul prezice destinația fără a ști ruta corespunzătoare acesteia.
	
	
		\subsection{Frecvența predicției de rute}
		De obicei, waypoint-urile furnizate de modul de predicție sunt transformate într-o rută ce folosește drumuri din hartă, fapt ce durează câteva secunde.
		Pentru a nu supraîncărca sistemul de navigație cu prea multe predicții, dezolvatorul poate seta timpul minim dintre două predicții. Acestă setare se poate efectua chiar în timpul rulării.
		
		\subsection{Numărul maxim de predicții}
		Pentru a putea suporta multiple platforme, modulul permite setarea numărului maxim de rute prezise ce apar deoadată. Acestă setare se poate efectua chiar în timpul rulării.
		
		\subsection{Prioritizarea predicțiilor}
		Există mai multe criterii pentru a stabili prioritatea unei rute. 
		
		\begin{table}[!h]
		\caption{Criterii de prioritizare pentru predicția bazată pe rute}
		\centering
		\begin{tabular}{ | m{2,7cm} | m{3,6cm} | m{3,22cm} | m{3,22cm} | m{1,4cm} | }
		\hline
		\textbf{Criteriu} & \textbf{Descriere} & \textbf{Min (0\% probabilitate)} & \textbf{Max (100\% probabilitate)} & \textbf{Pondere} \\ 
		\hline
		 Frecvența rutei & Numărul de utilizări al unei rute & Niciodată & Folosită de mai mult de 10 ori & 4 \\
		\hline
		 Frecvența rutei într-un anumit interval de timp & Numărul de utilizări al unei rute într-un anumit interval de timp (de la -1 oră la +2 ore) față de ora curentă & Niciodată & Folosită de mai mult de 10 ori & 1 \\
		\hline
		 Frecvența rutei într-o anumită zi & Numărul de utilizări al unei rute în ziua curentă din săptămână & Niciodată & Folosită de mai mult de 10 ori & 1 \\
		\hline
		 Frecvența rutei într-un anumit grup de zile & Numărul de utilizări al unei rute într-un anumit grup de zile (e.g. luni-vineri)& Niciodată &  Folosită de mai mult de 10 ori  & 1 \\
		\hline
		 Ultima utilizare a unei rute & Diferența dintre ultima utilizare a rutei și ora/data curentă & >4 săptămâni & <2zile & 3 \\
		\hline
		 Distanța până la destinație & Distanța dintre poziția actuală și destinație & >100km & 0km & 1 \\
		\hline
		\end{tabular}
		\label{table:tabel_predictii}
		\end{table}
		
		
		\subsection{Predicții bazate pe filtrarea rutei în funcție de distanța de la poziția curentă la waypoint-urile rutelor}
		Criteriile definite în tabela ~\ref{table:tabel_predictii}, ``Criterii de prioritizare pentru predicția bazată pe rute'' sunt extinse prin adăugarea următoarelor criterii:
		
		\begin{table}[!h]
		\caption{Criterii de prioritizare pentru predicția bazată pe rutele din jurul unei poziții}
		\centering
		\begin{tabular}{ | m{2,7cm} | m{3,6cm} | m{3,22cm} | m{3,22cm} | m{1,4cm} | }
		\hline
		\textbf{Criteriu} & \textbf{Descriere} & \textbf{Min (0\% probabilitate)} & \textbf{Max (100\% probabilitate)} & \textbf{Pondere} \\ 
		\hline
		 Distanța până la rută & Distanța dintre poziția actuală și waypoint-urile rutei &> 1km & 0km & 1 \\
		\hline
		 Direcția către rută & Diferența dintre orientarea waypoint-urilor și direcția de navigare & 180$^{\circ}$ & 0$^{\circ}$ & 1 \\
		\hline
		\end{tabular}
		\end{table}
		
		
		\subsection{Predicții bazate pe filtrarea rutei în funcție de distanța de la poziția curentă la punctele de start ale rutelor}
		Criteriile definite în tabela ~\ref{table:tabel_predictii}, ``Criterii de prioritizare pentru predicția bazată pe rute'' sunt extinse prin adăugarea următoarelor criterii:
		
		\begin{table}[!h]
		\caption{Criterii de prioritizare pentru predicția bazată pe filtrarea rutele în funcție de distanța până la punctul de start al rutei}
		\centering
		\begin{tabular}{ | m{2,7cm} | m{3,6cm} | m{3,22cm} | m{3,22cm} | m{1,4cm} | }
		\hline
		\textbf{Criteriu} & \textbf{Descriere} & \textbf{Min (0\% probabilitate)} & \textbf{Max (100\% probabilitate)} & \textbf{Pondere} \\ 
		\hline
		 Distanța până la punctul de start al rutei & Distanța până la cel mai apropiat punct de start al rutei &> 3km & 0km & 1 \\
		\hline
		\end{tabular}
		\end{table}
		
		\subsection{Predicții bazate pe filtrarea rutei în funcție de timpul scurs până la ajungerea la destinație}
		Pentru acest caz sunt folosite criteriile definite în tabela ~\ref{table:tabel_predictii}, ``Criterii de prioritizare pentru predicția bazată pe rute''.
		
		
		
		


	